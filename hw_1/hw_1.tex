\documentclass{article}
\usepackage{titling}
\usepackage{lipsum}
\usepackage{amsmath}
\usepackage{listings}
\usepackage{graphicx}
\usepackage{subcaption}



\begin{document}
\noindent
\begin{minipage}[t]{0.6\textwidth}
    \begin{flushleft}
        \LARGE\textbf{Math 343 - Homework 1} \\
        \vspace{6pt} % add 6pt of vertical space
        \hrule width 10cm
        \vspace{12pt}
        \large\textbf{Preston Duffield} \\
        \large Western Washington University \\
        % \today
        April 7, 2023
        \vspace{24pt}
    \end{flushleft}
\end{minipage}


\section*{Question 1}

\subsection*{a)}
I would choose to test if the two population means where equal, ie,
\begin{flushleft}
    $H_0$: $\mu_1 = \mu_2$ \\
    $H_a$: $\mu_1 \neq \mu_2$ \\
\end{flushleft}
\subsection*{b)}
\begin{equation*}
    \begin{array}{c|c|c}
        & \text{Location 1} & \text{Location 2} \\
        \hline
        \text{Sample Mean} & \bar{y}_1 = 2980 & \bar{y}_2 = 3205 \\
        \text{Sample Standard Deviation} & S_1 = 1140 & S_2 = 963 \\
        \text{Sample Size} & n_1 = 40 & n_2 = 40 \\
        \text{Sample Variance} & S_1^2 = 1299600 & S_2^2 = 927369 \\
    \end{array}
\end{equation*}\\

At significance level $\alpha = 0.10$. With degrees of freedom $v = (n_1 + n_2) - 2 = 40 + 40 - 2 = 78$.

\begin{align*}
    S_p &= \frac{(n_1 - 1)S_1^2 + (n_2 - 1)S_2^2}{n_1 + n_2 - 2} \\
    &= \frac{(40 - 1) \times 129960 + (40 - 1) \times 927369}{40 + 40 - 2} \\
    &= 528664.5
    \end{align*}

\begin{align*}
    t &= \frac{y_1-\bar{y_2}}{S_p \sqrt{\frac{1}{n_1}+\frac{1}{n_2}}} \\
    &= \frac{2980 - 3205}{528664.5 \sqrt{\frac{1}{40}+\frac{1}{40}}} \\
    &\approx -0.0019
    \end{align*}

The Test Statistic $t_{\alpha / 2 , v} = t_{.05 , 78} \approx 1.671$. Note that $t < t_{.05 , 78}$ and is not in the critical region.
Therefore, there is not enough statistical evidence to support the population means being unequal, that is, $\mu_1 \neq \mu_2$. 

\subsection*{c)}
The P-value can be obtained in R using the following code.
\begin{lstlisting}[language=R, caption=Calculating the P-value for a $t$ value, basicstyle=\small]
    # define the t-statistic
    t <- -0.0019
    
    # calculate the P-value for a two-tailed t-test
    p_value <- 2 * pt(t, df = 78, lower.tail = FALSE)
    
    # print the P-value
    p_value
    \end{lstlisting}

The following code shows that the P-value $= 1.001511$.
\subsection*{d)}
A $90\%$ confidence interval for the difference in the true mean distance between
the two populations is:
interval for the difference in the true mean distance between
\begin{align*}
    (y_1-\bar{y}2)&\pm t{\frac{\alpha}{2}, 78} * S_p * \sqrt{\frac{1}{n_1} + \frac{1}{n_2}} \\
    &\pm 1.671 * 528664.5 * \sqrt{\frac{1}{40} + \frac{1}{40}} \\
    &\pm 197533.8
    \end{align*}
\begin{equation*}
    (-197785.8, 197308.8)
\end{equation*}
This confidence interval contains zero, which is consistent with conclusion reported above.

\section*{Question 2}

First we note that,
% t = (y-bar/mu_0)/S/root(n) is approximately N(0,1).
\begin{equation*}
    t = \frac{\bar{X} - \mu_0}{S/\sqrt{n}} \sim N(0,1)
    \end{equation*}
\begin{flushleft}
Then we can derive that,
    \end{flushleft}
    
    % As long as the sample size is sufficiently large and the population is normal or the sample size is very large, the statement is generally true by the Central Limit Theorem.
\begin{align*}
    1 - \alpha &= P(-t_{\alpha,n-1} \leq \frac{\bar{X} - \mu_0}{S/\sqrt{n}} \leq t_{\alpha,n-1}) \\
                &= P(-t_{\alpha,n-1} \frac{S}{\sqrt{n}} \leq \bar{X} - \mu_0 \leq t_{\alpha,n-1} \frac{S}{\sqrt{n}}) \\
                &= P(-\bar{X} - t_{\alpha,n-1} \frac{S}{\sqrt{n}} \leq -\mu_0 \leq -\bar{X} + t_{\alpha,n-1} \frac{S}{\sqrt{n}}) \\
                &= P(\bar{X} + t_{\alpha,n-1} \frac{S}{\sqrt{n}} \geq \mu_0 \geq \bar{X} - t_{\alpha,n-1} \frac{S}{\sqrt{n}}) \\
                &= P(\bar{X} - t_{\alpha,n-1} \frac{S}{\sqrt{n}} \leq \mu_0 \leq \bar{X} + t_{\alpha,n-1} \frac{S}{\sqrt{n}})
    \end{align*}

Therefore, the confidence interval for one population mean $\mu$ in the case
where the population variance $\sigma ^{2}$ is unknown can be described as

\begin{align*}
    \bar{X} \pm t_{\alpha,n-1}\frac{S}{\sqrt{n}}
\end{align*}

\section*{Question 3}
The P-value can be obtained in R using the following code, where $t_0$ varies.
\begin{lstlisting}[language=R, caption=Calculating the P-value for a $t_0$ value, basicstyle=\small]
    # Define the t_0 value, degrees of freedom, and tail of the distribution
    t_0 <- 2.48
    df <- 10
    tail <- 2
    
    # Calculate the P-value using the "pt" function
    p_val <- pt(t_0, df, tail)
    
    # Print the P-value
    print(p_val)
    \end{lstlisting}

\subsection*{a)}
When $t_0 = 2.48$, the P-value is $0.637$.
\subsection*{b)}
When $t_0 = 3.55$, the P-value is $0.869$.
\subsection*{c)}
When $t_0 = 2.00$, the P-value is $0.478$.

\section*{Question 4}

\subsection*{a)}

\begin{figure}[h]
    \centering
    \includegraphics[width=0.5\textwidth]{./hw_1/images/4_b.png}
    \caption{The output of the 1-sample t test from Minitab.}
    \label{fig:4_a}
  \end{figure}
\begin{flushleft}
    $H_0$: The mean repair time is 225 hours. \\
    $H_a$: The mean repair time exceeds 225 hours.
\end{flushleft}
\subsection*{b/c)}
Since the p-value (0.257)
is greater than the significance level (0.05),
we fail to reject the null hypothesis. That is,
there is not enough statistical evidence to support that the mean repair
time exceeds 225 hours.
\subsection*{d)}
\begin{figure}[h]
    \centering
    \includegraphics[width=0.5\textwidth]{./hw_1/images/4_a.png}
    \caption{The Descriptive Statistics of the 1-sample t test from Minitab.}
    \label{fig:4_b}
  \end{figure}


\clearpage
\section*{Question 5}

\subsection*{a)}
\begin{figure}[h]
    \centering
    \includegraphics[width=0.5\textwidth]{./hw_1/images/5_a.png}
    \caption{Minitab output.}
    \label{fig:4_a}
\end{figure}
There is not enough statistical evidence to support the hypothesis that $\mu_1 = \mu_2$.
\subsection*{b)}
P-value $= 0.991$
\subsection*{c)}
\begin{figure}[h]
    \centering
    \includegraphics[width=0.5\textwidth]{./hw_1/images/5_b.png}
    \caption{Minitab output showing the confience interval.}
    \label{fig:4_a}
\end{figure}
we can be 95\% confident that the true difference between the two population means is somewhere between 2.475 and 4.117.
\clearpage
\subsection*{d)}
\begin{figure}[h]
    \centering
    \includegraphics[width=.7\textwidth]{./hw_1/images/5_c.png}
    \caption{Minitab output showing the dot plot.}
    \label{fig:4_a}
\end{figure}
\subsection*{e)}

\begin{figure}[h]
    \centering
    \begin{subfigure}[b]{0.4\textwidth}
        \includegraphics[width=1.25\textwidth]{./hw_1/images/5_d.png}
        \caption{Minitab output showing the normaility of 95 Celcius.}
      \label{fig:img1}
    \end{subfigure}
    \hfill
    \begin{subfigure}[b]{0.4\textwidth}
        \includegraphics[width=1.25\textwidth]{./hw_1/images/5_e.png}
        \caption{Minitab output showing the the normaility of 95 Celcius.}
      \label{fig:img2}
    \end{subfigure}
    \label{fig:both}
  \end{figure}

\paragraph{95} P-value = 0.161. There is not enough statistical evidence to reject the null hypothesis that the data comes from a normal distribution.
\paragraph{100} P-value = 0.457. There is not enough statistical evidence to reject the null hypothesis that the data comes from a normal distribution.

\clearpage
\section*{Question 6}

\subsection*{a)}
\begin{lstlisting}[language=R, caption=Calculating the P-value for a $t_0$ value, basicstyle=\small]
    > t.test(x_1, y = x_2, var.equal = F, conf.level = 0.95,
    +        alternative = "greater")
    
        Welch Two Sample t-test
    
    data:  x_1 and x_2
    t = -5.5372, df = 38.097, p-value = 1
    alternative hypothesis: true difference in means is greater than 0
    95 percent confidence interval:
     -4.037613       Inf
    sample estimates:
    mean of x mean of y 
     3.285714  6.380952 
    \end{lstlisting}
    There is enough statistical evidence to support that the true difference in means is greater than 0.
\subsection*{b)}
P-value $= 0.9999988$.
\subsection*{c)}
This confidence interval suggests that the true value of the difference
in means is likely to be greater than -4.037613.
\clearpage
\subsection*{d)}

  \begin{figure}[h]
    \centering
    \begin{subfigure}[b]{0.4\textwidth}
        \includegraphics[width=1.25\textwidth]{./hw_1/images/dot_1.png}
        \caption{Dot plot of 10 second cool-down time.}
      \label{fig:img1}
    \end{subfigure}
    \hfill
    \begin{subfigure}[b]{0.4\textwidth}
        \includegraphics[width=1.25\textwidth]{./hw_1/images/dot_2.png}
        \caption{Dot plot of 20 second cool-down time.}
      \label{fig:img2}
    \end{subfigure}
    \label{fig:both}
  \end{figure}

\subsection*{e)}
\begin{figure}[h]
    \centering
    \includegraphics[width=0.5\textwidth]{./hw_1/images/norm_1.png}
    \caption{Normal probability plot for 10 second cool-down time.}
    \label{fig:4_a}
  \end{figure}
  \begin{lstlisting}[language=R, caption=Shapiro-Wilk normality test for 10 second cool-down time., basicstyle=\small]
    > shapiro.test(x_1)

	Shapiro-Wilk normality test

    data:  x_1
    W = 0.89888, p-value = 0.03329
    \end{lstlisting}
    The 10 second cool-down time has a p-value of 0.03329, which is less than the significance level of 0.05. Therefore, we reject the null hypothesis and conclude that the sample data is not normally distributed.
  \begin{figure}[h]
    \centering
    \includegraphics[width=0.5\textwidth]{./hw_1/images/norm_2.png}
    \caption{Normal probability plot for 20 second cool-down time.}
    \label{fig:4_a}
  \end{figure}
  \begin{lstlisting}[language=R, caption=Shapiro-Wilk normality test for 20 second cool-down time., basicstyle=\small]
	Shapiro-Wilk normality test

    data:  x_2
    W = 0.93279, p-value = 0.1565
    \end{lstlisting}
    The 20 second cool-down time has a p-value of 0.1565, which is greater than the significance level of 0.05. Therefore, we reject the alternative hypothesis and conclude that the sample data is normally distributed.
\section*{Question 7}
We can estimate the sample mean by observing the values at the $50^{th}$ percentile. This gives us:
$\stackrel{\sim}{\mu}= 98.6$\\
We can estimate the sample standard deviation by taking the reciprocal of the slope of the best-fit line.
This can be approximated as:
$\stackrel{\sim}{\sigma} = 1/0.27 = 3.7$\\
\end{document}
